\section{Aspects légaux}\label{section:aspects_legaux}

% Module 1 %
\subsection{Introduction à la PI}
Tout d’abord, avant de trancher quel type de propriété intellectuelle (PI) est le mieux adapté pour notre projet, nous souhaitons donner une brève définition de la PI ainsi que présenter les différentes possibilités qui sont à dispositions et qui assurent la protection de notre projet.\newline

Le rôle principal de la propriété intellectuelle est la protection d’une nouvelle invention ; l’obtention d’un titre de PI donne l’exclusivité sur l’exploitation des nouvelles  découvertes et le droit d'empêcher les autres d'exploiter ces découvertes - ce qui permet en outre de limiter la concurrence dans son domaine, amortir sa R&D, augmenter ses marges ainsi que réduire les risques face aux brevets de concurrents.\newline

ll existe plusieurs types de titre de PI, parmi lesquels on retrouve principalement :

\paragraph{Les brevets :}
Il s'agit d'une forme de droit accordé par le gouvernement à un inventeur ou à son ayant cause, donnant au propriétaire le droit d'interdire à d'autres de fabriquer, d'utiliser, de vendre, d'offrir de vendre et d'importer une invention pendant une période limitée de temps, en échange de la divulgation publique de l'invention. Une invention est une solution à un problème technologique spécifique, qui peut être un produit ou un procédé, et doit généralement répondre à trois exigences principales : elle doit être nouvelle , non évidente et avoir une applicabilité industrielle .  Afin d'enrichir l'ensemble des connaissances et de stimuler l'innovation, les titulaires de brevets ont l'obligation de divulguer au public des informations précieuses sur leurs inventions ;

\paragraph{Les droits d'auteurs :} 
Ce sont eux qui confèrent au créateur d'une œuvre originale des droits exclusifs sur celle-ci, généralement pour une durée limitée. Le droit d'auteur peut s'appliquer à un large éventail de formes créatives, intellectuelles ou artistiques, ou œuvres.  Le droit d'auteur ne couvre pas les idées et les informations elles-mêmes, seulement la forme ou la manière dont elles sont exprimées ;

\paragraph{Les marques :} Une marque est un signe , un dessin ou une expression reconnaissable qui distingue les produits ou services d'un commerçant particulier des produits ou services similaires d'autres commerçants.\newline

Il est important de noter que a liste des PIs ci-dessus est loin d'être exhaustive; il existe d’autres titres tels que le désigne, ou encore le trade secret (secret commercial). On a cependant ici décidé d'exposer ceux qu'il peut être intéressant d'investiguer pour notre projet. 

\newpage

% Module 2 %
\subsection{Les brevets}
 
De sa définition informatique, une plateforme désigne un matériel ou un logiciel qui héberge une application ou un service. De ce fait, elle fait partie des exclusions de brevabilité. De plus, pour qu'un service puisse être brevetable, il doit remplir certains critères de brevabilité :\newline

 \begin{itemize}
     \item \textbf{L'utilisabilité industrielle :} ce critère est rempli par notre service car il pourra être utilisé par des entreprises différentes et pourra aider au recrutement de profils de différents corps de métier. \\
     Par contre, il ne respecte pas les critères de :\\
     \item \textbf{Nouveauté :} car il existe déjà - sur le marché Suisse et mondial - de nombreuses plateformes de recrutement basée sur l'IA.\\
     \item \textbf{Activité inventive :} Aucune activité inventive ne résulte de l'usage de cette plateforme. \\
 \end{itemize}
 
Seulement 1 critère sur 3 de brevetabilité étant respecté, il n'est donc pas possible de breveter notre service.

%
%Néanmoins, bien que non brevetable, le service que nous proposons tend à prendre une grande place sur la scène du recrutement. Que ce soit pour les petites ou pour les grandes entreprises, le but est d'effectuer un premier tri sur la panoplie de candidatures, nous le souligne \href{https://www.allnews.ch/content/points-de-vue/intelligence-artificielle-et-recrutement}{Allnews} dans ce article dedié à ce sujet.  \\ 
 %Sur le marché suisse, nous avons pu recenser quelques concurrents potentiels. Des entreprises qui proposent un service similaire au nôtre. On cite: \href{https://www.textkernel.com/fr/?utm_term=recrutement%20intelligence%20artificielle&utm_source=adwords&utm_campaign=FR+-+Search+-+Non+Brand&utm_medium=ppc&hsa_mt=p&hsa_src=g&hsa_kw=recrutement%20intelligence%20artificielle&hsa_ad=629928942899&hsa_tgt=kwd-541674329900&hsa_ver=3&hsa_acc=1393423699&hsa_net=adwords&hsa_cam=18670126951&hsa_grp=144320914202&gclid=CjwKCAiA5Y6eBhAbEiwA_2ZWISLyh3zdDh-KHgpPZ955DcROOiNeKEdHtwnNKpShXb0kbDMqB6jNvRoCTYwQAvD_BwE}{textKernel}, \href{https://www.hirevue.com/}{Hirevue} , \href{https://assessment.aon.com/en-us/video-interviewing-solution}{VidAssess} pour ne citer que ceux là. 

% Module 3 %
\subsection{Protection des logiciesls, autres stratégies de protection}
% --> Jbstand el bogoss

--> Faut utiliser les licenses pour pytorch, tensorflow, etc, décrire vitaif ces license

- Est-ce que le produit / le service de mon projet comprend du software ? Si cela est le cas, décrire les avantages / inconvénients de sa protection par le droit d’auteur.

- Si cela est le cas, décrire les avantages / inconvénients de sa protection par un brevet.

- Est-ce que certaines caractéristiques de mon projet pourraient être protégées par les trade secrets ? Si cela est le cas, décrire quelques mesures à mettre en place pour mettre en oeuvre cette protection.

- Est-ce que certaines caractéristiques de mon projet pourraient être divulguées via une publication défensive ? Si cela est le cas, décrire les avantages et inconvénients d’une telle stratégie.

% Module 4 %
\subsection{Liberté d'exploitation - Titularité}

Notre plateforme de recrutement inclut des technologies avancées telles que l’analyse de données, l’apprentissage automatique,  la mise en correspondance des profils et l’analyse des compétences. Elle gère également les données des candidats, comme les CV et l’évalutaion des compétences.\newline 

Après avoir effectué des recherches sur des bases de données de brevets comme \textit{Espacenet} et \textit{Swissreg}, nous avons découvert que certains domaines techniques de notre plateforme sont couverts par des brevets valides; en voici deux : 
•   WO2019068253A1 MACHINE LEARNING SYSTEM FOR JOB APPLICANT RESUME SORTING
•   FR2808354A1 PROCEDE DE MISE EN CONCORDANCE AUTOMATIQUE D'UNE OFFRE D'EMPLOI AVEC UNE PLURALITE DE PROFILS
Ces brevets couvrent l'apprentissage automatique pour le tri des CV de candidats ainsi que la mise en correspondance automatique entre une offre d'emploi et des profils de candidat.
Afin d’éviter une infraction de brevet, nous pouvons négocier une licence avec les propriétaires pour utiliser les éléments protégés dans notre service. 
Par ailleurs, si nous souhaitons développer notre projet en collaboration avec un partenaire académique ou un collaborateur externe, nous devons rédiger un contrat de partenariat. Ce contrat doit définir les droits d’utilisation, les obligations et les contributions de chacun-e des partenaires pour la publication ainsi que pour la divulgation du projet. 
En revanche, même si nous sommes seuls à avoir développé le projet et que nous allons fonder une start-up pour développer et commercialiser notre projet,  nous ne pouvons pas nous assurer être les propriétaires exclusifs de l'invention car nous ne disposons pas de brevet. 

% Module 5 %
\subsection{Contrats et litiges en matière de Propriété intellectuelle}

Il y a trois sortes de contrats les plus fréquents auxquels on peut s'intéresser pour que notre service remplisse et utilise les contraintes légales au mieux : les contrats de confidentialité, les contrats de collaboration/développement et les contrats de licence. Observons les donc ici et voyons qu'en est-il pour notre service : 

\paragraph{Le contrat de confidentialité} est un accord par lequel une ou plusieurs parties s'engagent à ne pas divulguer des informations considérées comme confidentielles. Il est souvent utilisé pour protéger les données sensibles, les secrets commerciaux ou les informations de propriété intellectuelle.

Ainsi, nous y mettrons les conditions cités en section \ref{subsection:protection_des_donnees} ; c'est-à-dire le fait que nous limiterons la conservation et le traitement de ces données  pour les besoins de la procédure de recrutement, qui prendra en général \textbf{moins de 6 mois}, ainsi que le fait que les données sensibles directes seront exclusivement utilisées pour l'identification du candidat, mais n'entreront pas en ligne de compte comme critère de sélection. Le type des données que cela concernera sera donc toutes les données présentes sur le CV du.de la candidat.e. Nous mettrons également dans ce contrat le fait que la récolte de CV se fait uniquement à des fins d'entraînement de modèle et de sélection de candidat.e.s, ainsi que la liste des corps de métiers de l'entreprise qui auront le droit d'y toucher, c'est-à-dire uniquement ceux qui vont travailler avec. \newline
Aussi, nous spécifierons dans ce contrat le fait que les données seront détruites après qu'un délai d'une année maximum ce soit écoulée depuis la récolte des données.

\paragraph{Le contrat de collaboration/développement} est un accord par lequel les parties s'engagent à travailler ensemble pour atteindre un objectif commun, généralement la création ou l'amélioration d'un produit ou d'un service. Il définit les responsabilités de chaque partie, les délais de développement, les coûts et les modalités de paiement.

Nous mettrons dans ce contrat le fait que la collaboration et le développement pour la construction du modèle de ML se fera seulement à l'interne, et que nul autre que les personnes travaillant sur ces données ou les consommant n'y aura accès, c'est-à-dire les data engineer / scientists et les personnes du secteur RH.\newline

\paragraph{Les contrats de licence} est un accord par lequel une partie (le titulaire des droits) autorise une autre partie (le licencié) à utiliser un bien protégé par des droits de propriété intellectuelle, comme un brevet, une marque déposée, une œuvre protégée par le droit d'auteur, ou un logiciel, en échange d'une rémunération ou de certains engagements. Il définit les droits d'utilisation, les restrictions d'utilisation, les coûts et les modalités de paiement, les clauses de confidentialité, les garanties et les procédures de résolution des conflits.

Nous mettrons dans ce contrat le fait que les entreprises utilisant notre service n'ont pas le droit de le monétiser pour d'autres entreprises et que toute transaction financière pour l'utilisation de ce service se fera uniquement des entreprises l'utilisant à notre entreprise. \newline

C'est aussi ici que l'on spécifiera que les personnes utilisant notre service auront donc accès aux CV séléctionnés et sont tenus de garder une confidentialité stricte les concernant ; seulement les personnes du secteur RH ont le droit de les consulter. On spécifiera ici aussi les modalités de paiements et les coûts, que l'on déterminera au préalable à l'aide d'une étude marketing.

À noter qu'il est également important de se rappeler que les lois sur la protection des données et les lois sur l'éthique de l'IA doivent être respectées dans le cadre de ces contrats.

%%%


% Module 6 %
\subsection{Protection des données}\label{subsection:protection_des_donnees}

 Comme toute plateforme de Hiring, nous allons faire face à la problématique de traitement des données utilisateurs. Étant donnée que nous nous limitons pour l'instant au marché Suisse comme mentionné en section \ref{section:description_projet}, nous nous appuyerons sur la loi \href{https://www.fedlex.admin.ch/eli/cc/1993/1945_1945_1945/fr}{LFPD (Loi Fédérale
sur la Protection des Données)}. Celle-ci régit le traitement de ces données afin de protéger au mieux la personnalité et les droits fondamentaux des personnes qui font l’objet d’un traitement de données en Suisse. \newline 
Nous nous baserons donc sur ses prescriptions en matière collecte, conservation, exploitation, modification,  communication, archivage ou destruction des données personnelles.\newline

Tout d'abord, une donnée personnelle peut être définie comme toute information se rapportant à une personne physique susceptible d'être identifiée ou identifiable, directement ou indirectement.  Durant le processus de recrutement via notre plateforme, les postulant.e.s nous mettrons à disposition - via leur CV - différentes données personnelles afin de permettre l'étude de leur dossier. Pour ce qui est des données personnelles que nous récoltons, on les catégorise en deux groupes :\newline

\begin{enumerate}
\item[-] \textbf{directes :} nom, prénom, photo, statut social ou marital ; \\
\item[-] \textbf{indirectes :} adresse postale, adresse e-mail, numéro de téléphone, date de naissance, études, formation professionnelle, nationalité, genre, langues parlées, formulaire de candidature ainsi que d'autres informations relatives à la candidature. Y compris la lettre de motivation, le CV avec - les expériences professionnelles antérieures - les qualifications professionnelles et autres compétences pertinentes - les références.\\
\end{enumerate}

Parmi ces données, aucune n'est à caractère sensible. Toutefois, afin de respecter au mieux la LPD nous devrons:\newline

\begin{itemize}
    \item Vérifier la véracité des données avant un quelconque traitement ;\\
    \item Tenir des postulant.e.s informé.e.s sur l'usage qui sera fait de leurs données ;\\
    \item Leur donner la possibilité de demander des informations portants sur l'utilisation de leurs données afin de garantir toute transparence ;\\
    \item Leur donner la possibilité de modifier ou de supprimer leur candidature ;\\
    \item Limiter les accès aux données personnelles récoltées : seules les personnes clés dans le processus pourront y accéder, entre autres les RH ;\\
    \item Les candidat.e.s seront immédiatement informé.e.s en cas de violation de leur données. \\
\end{itemize}

 La LPD ne prévoit pas de délai de conservation spécifiques pour les données ; cette durée est définie au cas par cas. Afin d'éviter d'éventuelles violations suite à une durée de conservation trop longue, nous limiterons la conservation et le traitement de ces données  pour les besoins de la procédure de recrutement, qui prendra en général \textbf{moins de 6 mois}. Les données sensibles directes seront exclusivement utilisées pour l'identification du candidat, mais n'entreront pas en ligne de compte comme critère de sélection ou non - comme discuté d'ailleurs également en analyse des aspects éthiques de notre service, en section \ref{section:aspects_ethiques} de ce rapport.\newline
 
En déposant sa candidature, le.la candidat.e à un emploi déclare avoir lu et compris notre politique en matière de confidentialité des données, laquelle expose la manière dont nous recueillons, traitons et utilisons ses données personnelles.