\section{Aspects légaux}\label{section:aspects_legaux}

%
% - Farouk  Module 1
% - @herycka_lulu_in_switzerland Module 2
% - @jbstand Module 3
% - Tegest Module 4
% - moi Module 5
%


% --
https://taleez.com/w/blog/le-parsing-cv-comment-ca-marche-et-pourquoi-lutiliser#title4
--> "Il doit respecter des critères déontologiques, et être conforme au RGPD (Règlement Général sur la Protection des Données), puisque des données liées aux candidats sont amenées à être conservées."
% --

Module 1 : 
--> attention, citer le fait que brevets pas applicables ici et justifier


Module 3 (license) : 
--> Faut utiliser les licenses pour pytorch, tensorflow, etc, décrire vitaif ces license

\subsection{Les brevets}
 
De sa définition informatique, une plateforme désigne un matériel ou un logiciel qui héberge une application ou un service. De ce fait elle fait partie des exclusions de brevabilité. De plus, pour qu'un service puisse être brevetable, il doit remplir certains critères de brevabilité : 
 \begin{itemize}
     \item L'utilisabilité industrielle : ce critère est remplie par notre service car il pourra être utilisé par des entreprises différentes et pourra aider au recrutement de profils de différents corps de métier. \\
     Par contre, il ne respecte pas les critères de :
     \item Nouveauté : car il existe déjà sur le marché Suisse et mondial, de nombreuses plateformes de recrutement basée sur l'IA.
     \item Activité inventive : Aucune activité inventive ne resulte de l'usage de cette plateforme. 
 \end{itemize}
Seulement 1 critère sur 3 de brevetabilité étant respecté, il n'est donc pas possible de breveter notre service. \\
Néanmoins, bien que non brevetable, le service que nous proposons tend à prendre une grande place sur la scène du recrutement. Que ce soit pour les petites ou pour les grandes entreprises, le but est d'effectuer un premier tri sur la panoplie de candidatures, nous le souligne \href{https://www.allnews.ch/content/points-de-vue/intelligence-artificielle-et-recrutement}{Allnews} dans ce article dedié à ce sujet.  \\ 
 Sur le marché suisse, nous avons pu recenser quelques concurrents potentiels. Des entreprises qui proposent un service similaire au nôtre. On cite: \href{https://www.textkernel.com/fr/?utm_term=recrutement%20intelligence%20artificielle&utm_source=adwords&utm_campaign=FR+-+Search+-+Non+Brand&utm_medium=ppc&hsa_mt=p&hsa_src=g&hsa_kw=recrutement%20intelligence%20artificielle&hsa_ad=629928942899&hsa_tgt=kwd-541674329900&hsa_ver=3&hsa_acc=1393423699&hsa_net=adwords&hsa_cam=18670126951&hsa_grp=144320914202&gclid=CjwKCAiA5Y6eBhAbEiwA_2ZWISLyh3zdDh-KHgpPZ955DcROOiNeKEdHtwnNKpShXb0kbDMqB6jNvRoCTYwQAvD_BwE}{textKernel}, \href{https://www.hirevue.com/}{Hirevue} , \href{https://assessment.aon.com/en-us/video-interviewing-solution}{VidAssess} pour ne citer que ceux là. 


\subsection{Protection des données}

Une donnée personnelle peut être définie comme toute information se rapportant à une personne physique susceptible d'être identifiée ou identifiable, directement ou indirectement.  Durant le procéssus de recrutement via notre plateforme, les postulants devront mettre à disposition différentes données personnelles ou non afin de permettre l'étude de leur dossier. Pour ce qui est des données personnelles, on catégorise ces données en deux groupes, les données personnelles:
\begin{itemize}
\item \textb{directes} : Nom, prénom, photo, [statut social ou marital]
\item \textb{indirectes} adresse postale, adresse email, numéro de téléphone, date de naissance, username,
\end{itemize}
