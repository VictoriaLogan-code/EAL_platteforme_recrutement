\section{Aspects légaux}\label{section:aspects_legaux}

%
% - Farouk  Module 1
% - @herycka_lulu_in_switzerland Module 2
% - @jbstand Module 3
% - Tegest Module 4
% - moi Module 5
%


% --
https://taleez.com/w/blog/le-parsing-cv-comment-ca-marche-et-pourquoi-lutiliser#title4
--> "Il doit respecter des critères déontologiques, et être conforme au RGPD (Règlement Général sur la Protection des Données), puisque des données liées aux candidats sont amenées à être conservées."
% --

 \subsection{Les brevets}
 
De sa définition informatique, une plateforme désigne un matériel ou un logiciel qui héberge une application ou un service. De ce fait elle fait partie des exclusions de brevabilité. De plus, pour qu'un service puisse être brevetable, il doit remplir certains critères de brevabilité : 
 \begin{itemize}
     \item L'utilisabilité industrielle : ce critère est remplie par notre service car il pourra être utilisé par des entreprises différentes et pourra aider au recrutement de profils de différents corps de métier. \\
     Par contre, il ne respecte pas les critères de :
     \item Nouveauté : car il existe déjà sur le marché Suisse et mondial, de nombreuses plateformes de recrutement basée sur l'IA.
     \item Activité inventive : Aucune activité inventive ne resulte de l'usage de cette plateforme. 
 \end{itemize}
Seulement 1 critère sur 3 de brevetabilité étant respecté, il n'est donc pas possible de breveter notre service. \\
Néanmoins, bien que non brevetable, le service que nous proposons tend à prendre une grande place sur la scène du recrutement. Que ce soit pour les petites ou pour les grandes entreprises, le but est d'effectuer un premier tri sur la panoplie de candidatures, nous le souligne \href{https://www.allnews.ch/content/points-de-vue/intelligence-artificielle-et-recrutement}{Allnews} dans ce article dedié à ce sujet.  \\ 
 Sur le marché suisse, nous avons pu recenser quelques concurrents potentiels. Des entreprises qui proposent un service similaire au nôtre. On cite: \href{https://www.textkernel.com/fr/?utm_term=recrutement%20intelligence%20artificielle&utm_source=adwords&utm_campaign=FR+-+Search+-+Non+Brand&utm_medium=ppc&hsa_mt=p&hsa_src=g&hsa_kw=recrutement%20intelligence%20artificielle&hsa_ad=629928942899&hsa_tgt=kwd-541674329900&hsa_ver=3&hsa_acc=1393423699&hsa_net=adwords&hsa_cam=18670126951&hsa_grp=144320914202&gclid=CjwKCAiA5Y6eBhAbEiwA_2ZWISLyh3zdDh-KHgpPZ955DcROOiNeKEdHtwnNKpShXb0kbDMqB6jNvRoCTYwQAvD_BwE}{textKernel}, \href{https://www.hirevue.com/}{Hirevue} , \href{https://assessment.aon.com/en-us/video-interviewing-solution}{VidAssess} pour ne citer que ceux là. 

 \begin{itemize}
    Notre plateforme de recrutement inclut des technologies avancées telles que l’analyse de données, l’apprentissage automatique, la recherche d’information, la mise en correspondance des profils et l’analyse des compétences. Elle intègre également des méthodes pour la gestion des données des candidats, comme la gestion des CV et l’évalutaion des compétences.
Après avoir effectué des recherches sur des bases de données de brevets comme Espacenet et Swissreg, nous avons constaté qu'il existe déjà des brevets dans certains domaines techniques de notre plateforme de recrutement. Ces brevets couvrent des aspects tels que l'apprentissage automatique pour le tri des CV de candidats et la mise en correspondance automatique entre une offre d'emploi et des profils de candidats.
•	WO2019068253A1 MACHINE LEARNING SYSTEM FOR JOB APPLICANT RESUME SORTING
•	FR2808354A1 PROCEDE DE MISE EN CONCORDANCE AUTOMATIQUE D'UNE OFFRE D'EMPLOI AVEC UNE PLURALITE DE PROFILS
Après avoir étudier ces  brevets, nous constatons qu’il existe plusieurs éléments similaires à notre plateforme, ce qui soulève des risques potentiels.
Tout d’abord, il peut s’agir d’une infraction de brevet. Comme les brevets mentionnés ci-dessus sont encore valide et que notre projet utilise des éléments protégés par ce brevet, nous pourrions être poursuivi en justice pour violation de brevet. 
Cependant, il est important de noter qu'il existe des moyens de contourner ces brevets ou de négocier une licence avec les propriétaires de ces brevets pour éviter toute violation potentielle. 
Par ailleurs, si nous souhaitons développer notre projet en collaboration avec un partenaire académique ou un collaborateur externe, il est essentiel de définir les termes de propriété intellectuelle dans un contrat de partenariat qui couvre les éléments créés tels que les brevets, les marques déposées, les dessins, les codes sources, les données, etc. Il est également important de définir les droits d'utilisation, les obligations et les contributions de chacun des partenaires, notamment pour la publication et la divulgation du projet.
En revanche, si nous sommes seuls à avoir développé le projet et que nous allons fonder une start-up pour développer et commercialiser notre projet, il est essentiel de prendre en compte les enjeux liés à la propriété intellectuelle. Pour protéger notre invention, il est important de déposer un brevet et de  respecter les démarches administrative. Il est également crucial de s'assurer que nous sommes les propriétaires exclusifs de l'invention protégée par le brevet. Pour déterminer qui a le droit d'exploiter commercialement l'invention, il est important de mettre en place une stratégie claire. Nous devons également décider si nous souhaitons accorder des licences à d'autres entreprises pour utiliser notre invention.
 En outre, il est important de protéger les informations confidentielles liées à notre invention, notamment les détails de la technologie et les données de test. Il faudrait également établir une stratégie de protection de la propriété intellectuelle à l'échelle nationale et internationale. Finalement, il est essentiel de définir les obligations en matière de propriété intellectuelle pour les employés, les partenaires et les investisseurs dans les contrats conclus avec eux.


\end{itemize}
