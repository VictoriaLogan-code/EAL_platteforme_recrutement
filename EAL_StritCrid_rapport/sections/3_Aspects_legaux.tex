\section{Aspects légaux}\label{section:aspects_legaux}

%
% - Farouk  Module 1
% - @jbstand Module 3
% - Tegest Module 4
% - moi Module 5
% - @herycka_lulu_in_switzerland Module 6
%


% --
https://taleez.com/w/blog/le-parsing-cv-comment-ca-marche-et-pourquoi-lutiliser#title4
--> "Il doit respecter des critères déontologiques, et être conforme au RGPD (Règlement Général sur la Protection des Données), puisque des données liées aux candidats sont amenées à être conservées."
% --

Module 1 : 
--> attention, citer le fait que brevets pas applicables ici et justifier


Module 3 (license) : 
--> Faut utiliser les licenses pour pytorch, tensorflow, etc, décrire vitaif ces license

% Module 1 %
\subsection{Introduction à la PI}
Tout d’abord avant de trancher quel type de propriété intellectuelle (PI) est le mieux adapté pour notre projet, nous souhaitons donner  une brève définition de la PI ainsi que présenter les  différentes possibilités qui sont à dispositions et qui assure la protection de notre projet.

Le rôle principal de la propriété intellectuelle est la protection d’une nouvelle invention, l’obtention d’un titre de PI donne l’exclusivité sur l’exploitation des nouvelles  découvertes et avoir le droit d'empêcher les autres d'exploiter ces découvertes, ce qui permet en outre de  limiter la concurrence dans son domaine, amortir sa R&D, augmenter ses marges ainsi que réduire les risques face aux brevets de concurrents.

ll existe plusieurs  types de titre de PI principalement on peut citer :  

\begin{enumerate}
    \item[-] \textbf{un brevet} est une forme de droit accordé par le gouvernement à un inventeur ou à son ayant cause, donnant au propriétaire le droit d'interdire à d'autres de fabriquer, d'utiliser, de vendre, d'offrir de vendre et d'importer une invention pendant une période limitée de temps, en échange de la divulgation publique de l'invention. Une invention est une solution à un problème technologique spécifique, qui peut être un produit ou un procédé, et doit généralement répondre à trois exigences principales : elle doit être nouvelle , non évidente et avoir une applicabilité industrielle .  Afin d'enrichir l'ensemble des connaissances et de stimuler l'innovation, les titulaires de brevets ont l'obligation de divulguer au public des informations précieuses sur leurs inventions ;
    
    \item[-] \textbf{un droit d'auteur} confère au créateur d'une œuvre originale des droits exclusifs sur celle-ci, généralement pour une durée limitée. Le droit d'auteur peut s'appliquer à un large éventail de formes créatives, intellectuelles ou artistiques, ou œuvres.  Le droit d'auteur ne couvre pas les idées et les informations elles-mêmes, seulement la forme ou la manière dont elles sont exprimées ;
    
    \item[-] \textbf{une marque} est un signe , un dessin ou une expression reconnaissable qui distingue les produits ou services d'un commerçant particulier des produits ou services similaires d'autres commerçants. 
\end{enumerate}

NB : la liste des PIs ci-dessus est loin d'être exhaustive,  il existe d’autres titres tel que  le désigne, le trade secret (secret commerciale) et bien d'autres formes, mais on a surtout exposer ceux qui semble intéressants pour notre projet. 


% Module 2 %
\subsection{Les brevets}
 
De sa définition informatique, une plateforme désigne un matériel ou un logiciel qui héberge une application ou un service. De ce fait elle fait partie des exclusions de brevabilité. De plus, pour qu'un service puisse être brevetable, il doit remplir certains critères de brevabilité : 
 \begin{itemize}
     \item L'utilisabilité industrielle : ce critère est remplie par notre service car il pourra être utilisé par des entreprises différentes et pourra aider au recrutement de profils de différents corps de métier. \\
     Par contre, il ne respecte pas les critères de :
     \item Nouveauté : car il existe déjà sur le marché Suisse et mondial, de nombreuses plateformes de recrutement basée sur l'IA.
     \item Activité inventive : Aucune activité inventive ne resulte de l'usage de cette plateforme. 
 \end{itemize}
Seulement 1 critère sur 3 de brevetabilité étant respecté, il n'est donc pas possible de breveter notre service. \\
Néanmoins, bien que non brevetable, le service que nous proposons tend à prendre une grande place sur la scène du recrutement. Que ce soit pour les petites ou pour les grandes entreprises, le but est d'effectuer un premier tri sur la panoplie de candidatures, nous le souligne \href{https://www.allnews.ch/content/points-de-vue/intelligence-artificielle-et-recrutement}{Allnews} dans ce article dedié à ce sujet.  \\ 
 Sur le marché suisse, nous avons pu recenser quelques concurrents potentiels. Des entreprises qui proposent un service similaire au nôtre. On cite: \href{https://www.textkernel.com/fr/?utm_term=recrutement%20intelligence%20artificielle&utm_source=adwords&utm_campaign=FR+-+Search+-+Non+Brand&utm_medium=ppc&hsa_mt=p&hsa_src=g&hsa_kw=recrutement%20intelligence%20artificielle&hsa_ad=629928942899&hsa_tgt=kwd-541674329900&hsa_ver=3&hsa_acc=1393423699&hsa_net=adwords&hsa_cam=18670126951&hsa_grp=144320914202&gclid=CjwKCAiA5Y6eBhAbEiwA_2ZWISLyh3zdDh-KHgpPZ955DcROOiNeKEdHtwnNKpShXb0kbDMqB6jNvRoCTYwQAvD_BwE}{textKernel}, \href{https://www.hirevue.com/}{Hirevue} , \href{https://assessment.aon.com/en-us/video-interviewing-solution}{VidAssess} pour ne citer que ceux là. 

% Module 3 %
\subsection{Protection des logiciesls, autres stratégies de protection}
% --> Jbstand el bogoss
- Est-ce que le produit / le service de mon projet comprend du software ? Si cela est le cas, décrire les avantages / inconvénients de sa protection par le droit d’auteur.

- Si cela est le cas, décrire les avantages / inconvénients de sa protection par un brevet.

- Est-ce que certaines caractéristiques de mon projet pourraient être protégées par les trade secrets ? Si cela est le cas, décrire quelques mesures à mettre en place pour mettre en oeuvre cette protection.

- Est-ce que certaines caractéristiques de mon projet pourraient être divulguées via une publication défensive ? Si cela est le cas, décrire les avantages et inconvénients d’une telle stratégie.

% Module 4 %
\subsection{Liberté d'exploitation - Titularité}

Notre plateforme de recrutement inclut des technologies avancées telles que l’analyse de données, l’apprentissage automatique, la recherche d’information, la mise en correspondance des profils et l’analyse des compétences. Elle intègre également des méthodes pour la gestion des données des candidats, comme la gestion des CV et l’évalutaion des compétences.
Après avoir effectué des recherches sur des bases de données de brevets comme Espacenet et Swissreg, nous avons constaté qu'il existe déjà des brevets dans certains domaines techniques de notre plateforme de recrutement. Ces brevets couvrent des aspects tels que l'apprentissage automatique pour le tri des CV de candidats et la mise en correspondance automatique entre une offre d'emploi et des profils de candidats.
•	WO2019068253A1 MACHINE LEARNING SYSTEM FOR JOB APPLICANT RESUME SORTING
•	FR2808354A1 PROCEDE DE MISE EN CONCORDANCE AUTOMATIQUE D'UNE OFFRE D'EMPLOI AVEC UNE PLURALITE DE PROFILS
Après avoir étudier ces  brevets, nous constatons qu’il existe plusieurs éléments similaires à notre plateforme, ce qui soulève des risques potentiels.
Tout d’abord, il peut s’agir d’une infraction de brevet. Comme les brevets mentionnés ci-dessus sont encore valide et que notre projet utilise des éléments protégés par ce brevet, nous pourrions être poursuivi en justice pour violation de brevet. 
Cependant, il est important de noter qu'il existe des moyens de contourner ces brevets ou de négocier une licence avec les propriétaires de ces brevets pour éviter toute violation potentielle. 
Par ailleurs, si nous souhaitons développer notre projet en collaboration avec un partenaire académique ou un collaborateur externe, il est essentiel de définir les termes de propriété intellectuelle dans un contrat de partenariat qui couvre les éléments créés tels que les brevets, les marques déposées, les dessins, les codes sources, les données, etc. Il est également important de définir les droits d'utilisation, les obligations et les contributions de chacun des partenaires, notamment pour la publication et la divulgation du projet.
En revanche, si nous sommes seuls à avoir développé le projet et que nous allons fonder une start-up pour développer et commercialiser notre projet, il est essentiel de prendre en compte les enjeux liés à la propriété intellectuelle. Pour protéger notre invention, il est important de déposer un brevet et de  respecter les démarches administrative. Il est également crucial de s'assurer que nous sommes les propriétaires exclusifs de l'invention protégée par le brevet. Pour déterminer qui a le droit d'exploiter commercialement l'invention, il est important de mettre en place une stratégie claire. Nous devons également décider si nous souhaitons accorder des licences à d'autres entreprises pour utiliser notre invention.
 En outre, il est important de protéger les informations confidentielles liées à notre invention, notamment les détails de la technologie et les données de test. Il faudrait également établir une stratégie de protection de la propriété intellectuelle à l'échelle nationale et internationale. Finalement, il est essentiel de définir les obligations en matière de propriété intellectuelle pour les employés, les partenaires et les investisseurs dans les contrats conclus avec eux.

% Module 5 %
\subsection{Contrats et litiges en matière de Propriété intellectuelle}

- Imaginez que vous avez fondé une start-up pour développer et commercialiser votre projet : décrire brièvement quels types de contrats vous permettraient de valoriser votre projet avec des tiers et essayez de décrire le « business model » que vous envisagez pour votre invention.


- Listez les avantages et inconvénients des différentes solutions légales par rapport à votre maîtrise sur le développement futur du produit et sa commercialisation.


- Imaginez d’avoir détecté une demande de brevet EP pending d’un concurrent très dangereuse pour votre projet. Décrivez les avantages / inconvénients de déposer des observations de tiers auprès de l’Office EP des Brevets.


- Imaginez d’avoir détecté un brevet EP d’un concurrent très dangereux pour votre projet. Décrivez les avantages / inconvénients d’une action en opposition auprès de l’Office EP des Brevets par rapport à une action en nullité auprès d’un Tribunal National dans un Pays dans lequel ce brevet EP a été validé.



% Module 6 %
\subsection{Protection des données}

 Comme toute plateforme de Hiring, nous allons faire face à la problématique de traitement des données utilisateurs. Étant donnée que nous nous limitons pour l'instant au marché Suisse, nous nous appuyerons sur la loi \href{https://www.fedlex.admin.ch/eli/cc/1993/1945_1945_1945/fr}{Loi fédérale
sur la protection des données}. Celle-ci régit le traitement de ces données afin de protéger au mieux la personnalité et les droits fondamentaux des personnes qui font l’objet d’un traitement de données en Suisse. Nous nous baserons donc sur ses prescriptions en matière collecte, conservation, exploitation, modification,  communication, archivage ou destruction des données personnelles.\\
Tout d'abord, une donnée personnelle peut être définie comme toute information se rapportant à une personne physique susceptible d'être identifiée ou identifiable, directement ou indirectement.  Durant le procéssus de recrutement via notre plateforme, les postulants devront mettre à disposition différentes données personnelles ou non afin de permettre l'étude de leur dossier. Pour ce qui est des données personnelles que nous récoltons, on les catégorise en deux groupes. Les données personnelles:
\begin{itemize}
\item \textbf{directes} : Nom, prénom, photo, [statut social ou marital]
\item \textbf{indirectes} adresse postale, adresse email, numéro de téléphone, date de naissance, études et formation professionnelle, nationalité, genre, langues parlées, formulaire de candidature ainsi que d'autres informations relatives à la candidature. Y compris la lettre de motivation, le CV avec - les expériences professionnelles antérieures - les qualifications professionnelles et autres compétences pertinentes - les références.
\end{itemize}
Parmi ces données, aucune n'est à caractère sensible. Toutefois, afin de respecter les au mieux la LPD nous devrons:
\begin{itemize}
    \item Vérifier la véracité des données avant un quelconque traitement.
    \item Tenir des postulants informé sur l'usage qui sera fait de leurs données.
    \item Afin de garantir toute transparence, leur donner la possibilité de demander des informations portants sur l'utilisation de leurs données.
    \item Leur donner la possibilité de modifier ou de supprimer leur candidature
    \item Limiter les accès aux données personnelles récoltées. Seuls les personnes clés dans le processus pourront y accéder, entre autres les RH.
    \item Les candidats seront immédiatement informés en cas de violation de leur données.
\end{itemize}

 La LPD ne prévoit pas de délais de conservation spécifiques pour les données (RH), cette durée est définie au cas par cas. Afin d'éviter d'éventuelles violations suite à une durée de conservation trop longue, nous limiterons la conservation et le traitement de ces données  pour les besoins de la procédure de recrutement. Celle-ci prend en général moins de 6 mois. Les données sensibles directes seront exclusivement utilisées pour l'identification du candidat, mais n'entreront pas en ligne de compte comme critère de sélection ou non. \\ 
En déposant sa candidature, le candidat à un emploi déclare avoir lu et compris notre politique en matière de confidentialité des données, laquelle expose la manière dont nous recueillons, traitons et utilisons ses données personnelles.