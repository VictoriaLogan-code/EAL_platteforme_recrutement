\section{Service étudié}\label{section:description_projet}

\subsection{Description du sujet}
Comme vu par certain.e.s d’entre nous en cours de Gouvernance des Données (GOD) le semestre passé (BA4), il existe des plateformes permettant l’aide au recrutement d’employé.e.s pour les entreprises. Ces plateformes se basent sur un modèle d'apprentissage automatique (machine learning) pour déterminer quel.le candidat.e il serait optimal d’engager. Nous allons donc ici nous intéresser à ce sujet en termes d’aspects éthiques et légaux en les analysant et en proposant des considérations et décisions basées sur ces aspects.

\subsection{Justification du choix du sujet}
Ce sujet nous intéresse particulièrement car la technologie et ses utilisations nous semblent se diriger vers cette direction ; en tant que futur.e employé.e, on devra probablement - selon nous - y faire face. De plus, étant en études dans le domaine de l’informatique, nous sommes conscient.e.s des failles et des limites de la digitalisation, comme les biais dans les algorithmes et dans les données qui peuvent fausser les résultats. Nous trouvons ainsi qu’il est pertinent d’approfondir le sujet pour en connaître d’avantage.

\subsection{Fonctionnement et utilisation du service}
Le service que nous avons imaginé étudier ici se base sur les services existants que nous avons trouvés lors de nos recherches en ligne, tels que \href{https://www.avrioai.com}{Avrio AI}, \href{https://www.manatal.com/?matchtype=e&adposition=&locphysicalid=1003191&device=c&utm_source=google&utm_medium=cpc&utm_term=ai%20recruiting%20software&utm_campaign=Switzerland_Search%20%7C%20Generic%20&%20Competitor%20Terms%20-%20ST&hsa_ver=3&hsa_ad=593873401673&hsa_cam=16975333135&hsa_tgt=kwd-382632061851&hsa_net=adwords&hsa_grp=135358698146&hsa_mt=e&hsa_src=g&hsa_kw=ai%20recruiting%20software&hsa_acc=9327528136&gclid=CjwKCAiAwomeBhBWEiwAM43YIBCeUxhnE7AhArtbkOJ4XJ5QOkSP0W2lle4zfltyFzJruynzU8IqUhoCOVwQAvD_BwE}{Manatal}, ou encore \href{https://www.workable.com}{Workable}.

Voici ainsi comment nous l'avons imaginé fonctionner : 

\begin{enumerate}
    \item[-] Chaque CV reçu est stocké dans une même base de donnée ;
    \item[-] Dans une première étape d'aide au recrutement, une analyse de CV est faite via un modèle de machine learning préentraîné. Ce modèle fournit des scores (note sur 10) pour les catégories qu'il a reperées, telles que les expériences du.de la candidat.e, pouvant ainsi par exemple donner un score de "management", ou encore de "années d'expérience dans le domaine". Pour donner un autre exemple de score fourni par le modèle, il classera également le niveau dans certain domaine, par exemple son niveau en C++, en Java, et autres languages de programmation. Ceci sera fait via le machine learning, qui fournira ces score suivant les phrases et leur formulation (contexte) trouvées dans le CV du.de la candidat.e ;
    \item[-] Dans une deuxième étape d'aide au recrutement, on utilisera un modèle de machine learning permettant de fournir un score de comparaison entre le profil d'un CV établi prédédemment et le profil recherché pour le poste en question.
\end{enumerate}

C'est sur ce fonctionnement général que nous allons nous baser pour analyser les aspects légaux (section \ref{section:aspects_legaux} de ce rapport) et les aspects éthiques (section \ref{section:aspects_ethiques} de ce rapport) inhérents.\newline

Finalement, il est important de préciser que nous avons imaginé cette application sous forme de service Suisse, c'est-à-dire que l'on simule ici le fait d'être une entreprise qui aimerait mettre ce service en place pour l'utilisation d'entreprises tierses Suisses.
